\enableregime[utf]
\usetypescript[modern][ec]
\setupbodyfont[10pt,rm]
\setupexternalfigures[location={local,global,default}]
\usemodule[t][markdown]
\def\markdownOptionHashEnumerators{true}
\def\markdownOptionDefinitionLists{true}
\def\markdownOptionSmartEllipses{true}
\def\markdownOptionFootnotes{true}
\def\markdownOptionInlineFootnotes{true}
\def\markdownOptionFencedCode{true}
\def\markdownOptionContentBlocks{true}
\definetyping [latex]
\setuptyping  [latex] [option=TEX]
\starttext
% Typeset the document `example.md` by letting the Markdown package handle
% the conversion internally.
\markdownInput{./example.md}

% Typeset the document `example.tex` that we prepared separately using the
% Lua command-line interface and that contains a plain TeX representation
% of the document `example.md`.
\doiffileelse{./example.tex}{\input example}{}

\startmarkdown
Here are some non-ASCII characters: *ěščřžýáíé*
and ConTeXt special characters: *|*.
\stopmarkdown
\stoptext
